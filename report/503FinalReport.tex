\documentclass[11pt, oneside]{article}   	% use "amsart" instead of "article" for AMSLaTeX format
\usepackage[margin=1in]{geometry}                		% See geometry.pdf to learn the layout options. There are lots.
\geometry{letterpaper}                   		% ... or a4paper or a5paper or ... 
%\geometry{landscape}                		% Activate for rotated page geometry
\usepackage[parfill]{parskip}    		% Activate to begin paragraphs with an empty line rather than an indent
\usepackage{graphicx}				% Use pdf, png, jpg, or eps§ with pdflatex; use eps in DVI mode
								% TeX will automatically convert eps --> pdf in pdflatex		
\usepackage{amssymb}
\usepackage{etoolbox}
    \AfterEndEnvironment{hangparas}{\addvspace{0.67\baselineskip}}
    \usepackage[notquote]{hanging}


\title{Here's What (and Whom) to Avoid When Driving}
\author{Desmond Cole, Teerth Patel, Yunbin Peng}

\begin{document}
\maketitle
\section*{Introduction}
We analyze traffic fatality data provided by the National Highway Traffic Safety Administration (NHTSA) to assess various predictors of traffic fatalities and develop a limited profile of the circumstances associated with traffic fatalities.

\section*{Exploratory Analysis and Visualization}

\subsection*{Geographic Patterns}

\subsection*{Time Trends}
\textbf{Daily Cycle} \\
At the national and state level, the cycle of fatal accidents throughout the day is fairly consistent. There is a local maximum in the early morning, correlating with morning rush hour. Beginning just before noon, the level of fatal accidents rises consistently to peak at between 7 and 8PM, before declining steadily through to about 3 or 4 in the morning. \\
\\
(Insert National graphic)
\\

With State-by-State plots, the daily fatal accident cycle is roughly similar, without significant change. The below plots show results from shape-based time series clustering of different states, according to daily fatal accident patterns. \\

(Insert side-by-side state-level centroid plots)

\textbf{Weekly Cycle} \\

\section*{Selected Predictors of Traffic Fatalities}

\subsection*{Driver Behavior}
\textbf{Drugs/Alcohol}
\textbf{Distraction}

\subsection*{Car Manufacturer}

\subsection*{Environmental Conditions}

\section*{Conclusion}

\section*{References}
\begin{hangparas}{1.27cm}{1}
Batterman, Stuart, Richard Cook, and Thomas Justin. "Temporal variation of traffic on highways and the development of accurate temporal allocation factors for air pollution analyses." \textit{Atmos Environ}. Apr. 2015.

Zador, P.L, S.A. Krawchuk, and R.B. Voas. "Relative Risk of Fatal Crash Involvement by BAC, Age, and Gender." \textit{U.S. Department of Transportation - National Highway Traffic Safety Administration}. Apr. 2000.




\end{hangparas}




\end{document}  