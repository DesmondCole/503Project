\documentclass[11pt, oneside]{article}   	% use "amsart" instead of "article" for AMSLaTeX format
\usepackage[margin=1in]{geometry}                		% See geometry.pdf to learn the layout options. There are lots.
\geometry{letterpaper}                   		% ... or a4paper or a5paper or ... 
%\geometry{landscape}                		% Activate for rotated page geometry
\usepackage[parfill]{parskip}    		% Activate to begin paragraphs with an empty line rather than an indent
\usepackage{graphicx}				% Use pdf, png, jpg, or eps§ with pdflatex; use eps in DVI mode
								% TeX will automatically convert eps --> pdf in pdflatex		
\usepackage{amssymb}
\usepackage{etoolbox}
    \AfterEndEnvironment{hangparas}{\addvspace{0.67\baselineskip}}
    \usepackage[notquote]{hanging}


\title{Here's What (and Whom) to Avoid When Driving}
\author{Desmond Cole, Teerth Patel, Yunbin Peng}

\begin{document}
\maketitle
\section*{Introduction}
We analyze traffic fatality data provided by the National Highway Traffic Safety Administration (NHTSA) to assess various predictors of traffic fatalities and develop a limited profile of the circumstances associated with traffic fatalities.

\section*{Exploratory Analysis and Visualization}

\subsection*{Geographic Patterns}

\subsection*{Time Trends}
\textbf{Daily Cycle} \\
At the national and state level, the cycle of fatal accidents throughout the day is fairly consistent. There is a local maximum in the early morning, correlating with morning rush hour. Beginning just before noon, the level of fatal accidents rises consistently to peak at between 7 and 8PM, before declining steadily through to about 3 or 4 in the morning. \\
\\
(Insert National graphic)
\\

With State-by-State plots, the daily fatal accident cycle is roughly similar, without significant change. The below plots show results from shape-based time series clustering of different states, according to daily fatal accident patterns. \\

(Insert side-by-side state-level centroid plots)

\textbf{Weekly Cycle} \\

\section*{Selected Predictors of Traffic Fatalities}
We used a mix of classifiers to assess factors relevant to incident fatalities, with a particular focus on driver behavior, environmental conditions, and vehicle manufacturer. In many cases, we had to grapple with significant class imbalance. For example, in predicting whether an accident results in multiple fatalities

\subsection*{Driver Behavior}
\textbf{Drugs/Alcohol}
\textbf{Distraction}

\subsection*{Car Type}

This section explores in some detail different fatality rates for different car types (by automaker, vehicle body, etc.).\footnote{Note that the numbers in this assessment focus on car brands \textit{involved} in fatal incidents. Thus, they do not imply specifically that the vehicle types considered here actually caused death(s), or that the driver(s) of the vehicle(s) themselves died.} For this assessment, we focused on the 15 largest automakers that together produce more than 95\% of the cars and light trucks on American roads. In addition, we subsetted the data to focus specifically on smaller vehicles, excluding commercial vehicles, semi-trucks, etc. \\

The plot below shows a basic ranking of car manufacturers according to the number of fatalities per million vehicles.

(Insert graph of fatalities/fatal accidents by car manufacturer) \\


The plot above, although suggestive of meaningful difference across manufacturers, fails to fully account for the various contextual differences which may be unobservable. To further explore the relevance of car manufacturer, we consider the relationships between car model and various fatality-related predictors, to understand if a given manufacturers' products tend to be associated with high-risk behaviors or other crash factors. The table below shows the results from various multi-class classifications of car manufacturer ran using a set of crash-relevant predictors .... The results do not suggest any substantial mechanism(s) by which a car's make determines the risk of it being involved in a fatal incident. \\

(Insert table of classifications and results) \\

As an alternative to classification, we applied multidimensional scaling to assess visually the differences between the major automakers in terms of various crash-relevant pieces of information. These results suggest some moderate clustering of driver behavior. BMW and Jaguar-Land Rover, both of which are luxury brands, are spaced relatively far from more truck-oriented manufacturers such as GM and Ford. In addition, Mitsubishi, which as a brand appears to include many heavier-duty vehicles, is an outlier from the majority of the other automakers.\\

(Insert MDS plot) \\



\subsection*{Environmental Conditions}

\section*{Conclusion}

\section*{References}
\begin{hangparas}{1.27cm}{1}
Batterman, Stuart, Richard Cook, and Thomas Justin. "Temporal variation of traffic on highways and the development of accurate temporal allocation factors for air pollution analyses." \textit{Atmos Environ}. Apr. 2015.

Choi, Eun-Ha. "Crash Factors in Intersection-Related Crashes: An On-Scene Perspective." \textit{U.S. Department of Transportation - National Highway Traffic Safety Administration}. Sept. 2010.

Zador, P.L, S.A. Krawchuk, and R.B. Voas. "Relative Risk of Fatal Crash Involvement by BAC, Age, and Gender." \textit{U.S. Department of Transportation - National Highway Traffic Safety Administration}. Apr. 2000.




\end{hangparas}




\end{document}  